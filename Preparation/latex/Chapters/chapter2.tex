%!TEX root = ../template.tex
%%%%%%%%%%%%%%%%%%%%%%%%%%%%%%%%%%%%%%%%%%%%%%%%%%%%%%%%%%%%%%%%%%%%
%% chapter2.tex
%% NOVA thesis document file
%%
%% Chapter with the template manual
%%%%%%%%%%%%%%%%%%%%%%%%%%%%%%%%%%%%%%%%%%%%%%%%%%%%%%%%%%%%%%%%%%%%

\typeout{NT FILE chapter2.tex}%

\chapter{NOVAthesis Template \emph{User's Manual}}
\label{cha:users_manual}

\glsresetall

\begin{center}
  \fbox{\LARGE
    This manual is outdated and must be revised!}
\end{center}


\section{Introduction}
\label{sec:introduction}


\section{Quick Start}
\label{sec:quick_started}

\subsection{With a Local \LaTeX\ Installation} % (fold)
\label{sub:with_a_local_latex_installation}

Follow these steps to get started with a local \LaTeX\ installation:

% subsection with_a_local_latex_installation (end)

\subsection{With a Remote Cloud-based Service} % (fold)
\label{sub:with_a_remote_cloud_based_service}


\section{Example glossary, acronyms, and symbols}
%
% \todo[inline]{A a note in a line by itself.}
%

%
% Please note that
% \begin{center}
%   \textbf{\large this package and template are not official for FCT/NOVA}.
% \end{center}



% \printbibliography[heading=subbibliography, segment=\therefsegment, title={\bibname\ for chapter~\thechapter}]

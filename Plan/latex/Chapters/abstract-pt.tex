%!TEX root = ../template.tex
%%%%%%%%%%%%%%%%%%%%%%%%%%%%%%%%%%%%%%%%%%%%%%%%%%%%%%%%%%%%%%%%%%%%
%% abstract-pt.tex
%% NOVA thesis document file
%%
%% Abstract in Portuguese
%%%%%%%%%%%%%%%%%%%%%%%%%%%%%%%%%%%%%%%%%%%%%%%%%%%%%%%%%%%%%%%%%%%%

\typeout{NT FILE abstract-pt.tex}%


Lógica é um tópico essencial em áreas como matemática e ciência da computação, sendo uma disciplina chave no currículo de informática. Neste contexto, alunos e professores carecem de ferramentas para complementar as aulas, especialmente para praticar, pois nem sempre os alunos podem interagir com os professores para esclarecer dúvidas.

Os cursos online são um bom exemplo de ferramentas que oferecem recursos acessíveis a todos. No entanto, no campo da lógica, estes cursos tipicamente não permitem a adição de novos materiais e são limitados no número de tipos de exercícios. A lógica inclui diversos tipos de exercícios, sendo a dedução natural sendo o mais desafiador. Infelizmente, não existem ferramentas disponíveis com esses exercícios. Além disso, muitas ferramentas carecem de mecanismos eficazes de feedback, deixando os utilizadores perdidos.

Desenvolver um sistema de feedback eficaz é essencial para alcançar uma ferramenta de sucesso. No entanto, é uma tarefa desafiante, pois não queremos um sistema onde os alunos dependam do feedback, mas ao mesmo tempo, não queremos que percam o interesse em aprender.

Nesta tese, propomos o desenvolvimento de uma ferramenta online interativa com o objetivo de ajudar os alunos a praticar exercícios de lógica. Queremos um sistema que seja acessível a todos, desde o utilizador novato em lógica até o utilizador experiente. O nosso foco principal será criar um sistema de feedback eficaz que guiará os alunos ao longo da construção das provas. Além disso, pretendemos fornecer mecanismos para adicionar novos exercícios e avaliá-los. Esta ferramenta suportará exercícios de dedução natural, sendo desenhada para permitir futuras expansões, incluindo outros tipos de exercícios. Por fim, integraremos a nossa ferramenta com uma plataforma de e-learning online, como o Moodle, uma vez que oferece mecanismos para gerir aulas e notas.

Com este projeto, pretendemos que os alunos tenham uma ferramenta de estudo acessível a todos, capaz de ajudá-los a superar, de uma forma eficiente e envolvente, os desafios que enfrentam nos exercícios de lógica.

% Palavras-chave do resumo em Português
% \begin{keywords}
% Palavra-chave 1, Palavra-chave 2, Palavra-chave 3, Palavra-chave 4
% \end{keywords}
\keywords{
  Lógica \and
  Sistemas Dedutivos \and
  Feedback \and
  Ferramentas Interativas \and
  Dedução Natural \and
  Lógica Proposicional \and
  Lógica de Primeira Ordem
}
% to add an extra black line

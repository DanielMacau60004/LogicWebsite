%!TEX root = ../template.tex
%%%%%%%%%%%%%%%%%%%%%%%%%%%%%%%%%%%%%%%%%%%%%%%%%%%%%%%%%%%%%%%%%%%
%% chapter1.tex
%% NOVA thesis document file
%%
%% Chapter with introduction
%%%%%%%%%%%%%%%%%%%%%%%%%%%%%%%%%%%%%%%%%%%%%%%%%%%%%%%%%%%%%%%%%%%

\typeout{NT FILE chapter4.tex}%

\chapter{Proposed Work}

Erros que podem ser reportados:
- syntactic errors expressions
    - Indicar que a expressão x está errada
    - Indicar a posição do error
    - Indicar a posição do erro e possiveis alternativas
- syntactic errors deductions
    - Indicar que a arvore nao esta a usar regras validas
    - Indicar que a regra x nao pode ser colocada depois da regra y
    - Indicar que o numero de ramificacoes nao é o correto, dizendo o valor esperado
    - Mostrar uma possivel solucao para a regra errada
    - Introdução de marcas invalidas (usar marca 100 quando so existem marcas ate 2)
- semantic errors deductions
    - Indicar que a expressao esta mal colocada na regra (colocar uma implicacao numa disjução lado esquerdo)
    - Indicar que existem conclusoes que nao foram bem deduzidas (a conclusao nao corresponde ao que deduzimos)
    - Dizer em concreto aquilo que está a ser deduzido
    - Associação de marcas invalidas (chamar marca 1 quando se queria chamar marca 2)
    - Numero de marcas associadas a uma regra é invalido
    - Indicar marcas que deveriam estar fechadas e nao estão (de forma concreta enumerando ou dizendo genericamente)
    - Indifcar marcas que deveriam estar na arvore (por exemplo na intro da neg, a hipotese tem que aparecer na arvore)

Feedback:
    - Indicar em concreto que parte da arvore está errdada (alterando a cor de fundo)
    - Mostrar que expressões de arvores diferentes podem ser juntadas
    - Automaticamente adicionar as regras já com o numero de ramificações corretas (nivel1)
    - Automaticamente adicionar as regras já com o numero de ramificacoes corretas e os campos preenchidos (nivel2)
    - Indicar a cada passo, as hipoteses que temos (nivel 1)
    - Indicar a cada passo, as hipoteses que temos e as hipoteses que devem aparecer na arvore mas que ainda nao aparecem (nivel 2)

Advanced Feedback:
    - Mostrar os passos a seguir / Apresentar caminhos alternativos quando o utilizador está preso (feed forward)
    - Indicar a quantos passos estamos de uma solução
    - Apresentar melhorias da solucao encontrada pelo aluno, por exemplo uma arvore mais pequena mas que usa a mesma sequencia de regras
    - Apresentar opcoes de subgoals a atingir antes de provar a conclusao, por exemplo prova primeiro X

The aim of this thesis is to design and implement a variety of types of exercises [REF] commonly found in introductory logic courses, both in PL and FOL. To enhance the students' engagement, the exercises should be presented in an interactive way. This should be supported by a feedback system that will guide students through the resolution of the exercises and help prevent them from getting lost. Developing a successful feedback system is not an easy task. We need to find a balanced way to provide the right amount of feedback without leaving the student even more lost in the exercise. This can depend on many factors, for example, the level of expertise of the student, the type of exercise, the resolution path that the student is considering, etc. We described some excellent examples of feedback systems in [CHAPTER 3]. This tool must also provide teachers with an intuitive way to add new exercises as well as a way to grade them. The purpose of this is to enable automatic evaluation through integration with existing online e-learning platforms like Moodle.

\section{Exercises}

There is a vast variety of exercises in logic. Below is a list of some exercises, each one followed with a brief description of how they work. %and, in some cases, how we can provide feedback, hints, and grade them.

\begin{itemize}
    \item \textbf{Transforming a sentence from natural language to PL/FOL:} This type of exercise essentially involves translating a declarative sentence in natural language into PL/FOL. For example, consider the following propositions:\[p: \text{It is raining}, \quad q: \text{It is cold}.\]
    Now, imagine the exercise asks you to write "If it is cold then it is raining" in PL, using the propositions defined above. A correct answer would be: \(q \rightarrow p\) or \(\neg q \lor p\). While developing this exercise, it is essential to consider that a question may have multiple answers, so an equivalence checking system is needed. Implementing an equivalence checking system seems pretty straightforward in PL, but when dealing with FOL expressions, the problem turns out to be extremely hard, since there's no algorithm that can always prove the equivalence of two different expressions[REF].
    \item \textbf{Build Truth Tables:} This exercise involves filling in the gaps of a truth table. The user is presented with an expression in PL, and the goal is to determine the final truth values by considering all possible combinations of truth values for the propositions. To simplify the exercise, the user can break the initial formula into smaller subformulas and, at the end, combine their values to compute the final truth value.
    \item \textbf{Conversion to Conjunctive, Disjunctive, and Negation Normal Forms:} In this exercise, students receive an expression and must transform it into various forms. For example, consider this expression:\[ (\varphi \land \psi) \lor \neg \theta \]
    The exercise asks you to convert this expression into CNF by distributing disjunctions over conjunctions, resulting in: \( (\varphi \lor \neg \theta) \land (\psi \lor \neg \theta) \). Alternatively, converting to DNF involves distributing conjunctions over disjunctions, resulting in: \( (\varphi \land \psi) \lor \neg \theta \).
    %\item \textbf{Conversion to Horn clauses:} This exercise entails the sequential application of a series of transformations, such as the elimination of implications, conversion to negation normal form, and Skolemization, until the final list of clauses is obtained. This type of exercise is useful in preparing students for practical applications, such as in Prolog, where Horn clauses are frequently used for efficient inference, and in practicing logical manipulations.
    \item \textbf{Natural deduction in tree shape:} In this exercise, students are required to prove the validity of an expression both in PL and FOL via natural deduction. As described in [REF], these proofs are based on established rules that can be used to infer new results that will help reach a conclusion. A detailed step-by-step resolution of these kinds of exercises is provided in [REF]. This is the type of exercise where students struggle the most, so it would be important to consider it when designing our system.
\end{itemize}

\subsection{Ambition Levels}
\subsection{Evaluation Plan}


\section{Work Plan}
%...
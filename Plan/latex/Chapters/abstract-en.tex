%!TEX root = ../template.tex
%%%%%%%%%%%%%%%%%%%%%%%%%%%%%%%%%%%%%%%%%%%%%%%%%%%%%%%%%%%%%%%%%%%%
%% abstract-en.tex
%% NOVA thesis document file
%%
%% Abstract in English([^%]*)
%%%%%%%%%%%%%%%%%%%%%%%%%%%%%%%%%%%%%%%%%%%%%%%%%%%%%%%%%%%%%%%%%%%%

\typeout{NT FILE abstract-en.tex}%

Logic is a fundamental topic in areas like mathematics and computer science and is a key discipline in the computer science curriculum. In this context, students and teachers often lack tools to complement logic classes, especially for practicing exercises, as it is not always possible for students to interact with teachers to address their difficulties.

Online courses are a good example of tools that provide accessible resources to everyone. However, in the field of logic, these courses typically do not allow for the addition of new material and are often limited in the number of exercises. Logic includes various exercises, with natural deduction posing the greatest challenges for students. Unfortunately, there are no tools available to support these exercises. Additionally, many of these tools lack effective feedback mechanisms, leaving users feeling lost.

Developing an effective feedback system is essential to achieving a successful tool. However, it is a challenging task, as we do not want a system where students are always dependent on the feedback and stop thinking by themselves, but at the same time, we do not want them to lose interest in learning because they are always stuck in exercises.

In this thesis, we propose the development of an interactive online tool whose goal is to help students practice logic exercises. We want a system that is meant for everyone, from the novice user that is starting their journey in logic to the experienced user that wants to improve even more their skills. Our primary focus will be on creating an effective feedback system that will guide students throughout the proofs. Additionally, we aim to provide mechanisms for adding new exercises and evaluating them. This tool will support natural deduction exercises, and it will be designed to allow for future expansions to include other types of exercises. Finally, we will integrate our tool with an online e-learning platform such as Moodle, as it provides mechanisms to manage classes and grades.

With this project, we want students to have a studying tool that is accessible to everyone and capable of assisting them in overcoming, in an efficient and engaging way, the challenges they face in logic exercises.

% Palavras-chave do resumo em Inglês
% \begin{keywords}
% Keyword 1, Keyword 2, Keyword 3, Keyword 4, Keyword 5, Keyword 6, Keyword 7, Keyword 8, Keyword 9
% \end{keywords}
\keywords{
  Logic \and
  Deductive System \and
  Feedback \and
  Interactive tool \and
  Natural Deduction \and
  Propositional Logic \and
  First-Order Logic
}
